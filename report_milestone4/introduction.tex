\section{Introduction}
Climate change is an issue of ever increasing significance to both policy makers and the public, especially as the impact of climate change on global and local economies has become clear. People care about this critical issue, and this interest is reflected in news headlines and across global social media.

Geologists and climatologists have developed many traditional methods the get more insight of this problem. They look into evidences from temperature measurements and proxies, historical and archaeological evidence, glaciers, arctic sea ice loss, vegetation, precipitation, sea level change and so on. They have also built many satellites and monitoring stations that are collecting large volumes of data everyday to help with the analysis. However, these traditional methods are mostly based on hypothesis testing methods and can't make full use of these automatically collected data, as the data volume is too big to handle. It is natural that novel methods that are able to deal with big data can play an important role in the climate change research.

We developed a system to cluster and visualize the GHCN-D data\cite{GHCN-D}.
GHCN-D is a dataset that contains daily weather observations over global land areas.
Like its monthly counterpart, GHCN-Daily is a composite of climate records from
numerous sources that were merged together and subjected to a common suite of quality assurance reviews. The archive includes the following meteorological elements: daily maximum temperature, daily minimum temperature, temperature at the time of observation, precipitation (i.e., rain, melted snow), snowfall, snow depth, other elements where available. The data we used dates from 1763 to 2014 and has approximately 100 Gigabytes total volume.

Our approach is a practical attempt to cluster, visualize and interpret the data, which brings insights into the change and may help to control negative effect.Further researches can be related to the patterns appeared in our result.


%Why is this problem interesting or useful?
%Why do we need a new system?
%Which alternative/traditional techniques can be used to solve the problem

%What is the main new contribution of your approach
%How does your solution compare to existing work


