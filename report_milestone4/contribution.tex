\section{Contribution}
Using big data tools to study climate change is getting more and more attention. In May of this year(2014), the United Nations kicked off a new initiative on climate change - the Big Data Climate Challenge\cite{climatechallenge}. One of the primary aims of the challenge is to use big data to make the case for climate change action, specifically ``to bring forward data-driven evidence of the economic dimensions of climate change". Also, NASA Center for Climate Simulation has developed a supercomputer called Discover to provide advanced, usable, agile and efficient high-performance computing services for tasks like global climate simulations at extremely high resolutions for long periods of time.\cite{nasa}. However, their work is based on the supercomputer and their technology details are not published. 

Our approach is a practical attempt to cluster, visualize and interpret the data, which brings insights into the change and may help to control negative effect. Further researches can be related to the patterns appeared in our result. One important idea of our project is that we did clustering on all data points from 1763 to 2014. We assume that the points that are spatiotemporal close to each other should be grouped into one cluster. If the label of monitoring stations in one area has changed, we may find the climate change in that area. This is different from the existing projects that are mainly focused on large-scale simulation and is our main contribution. 

%我觉得这部分单列出来比较蛋疼,没啥内容,可以和第一部分合在一起,这部分可以叫related work之类的


