\section{Data Sources and Data Preparation}

\begin{figure}[H]
\centering
\includegraphics[width=.85\linewidth]{./figure/Full_view.png}
	\caption{Visualization of climate pattern clustering with our system.}
	\label{fig:FullView}
\end{figure}

\subsection{Introduction}
GHCN-D is a dataset that contains daily observations over global land areas. 
Like its monthly counterpart, GHCN-Daily is a composite of climate records from 
numerous sources that were merged together and subjected to a common suite of quality 
assurance reviews. The archive includes the following meteorological elements:
\begin{itemize}
    \item Daily maximum temperature
    \item Daily minimum temperature
    \item Temperature at the time of observation
    \item Precipitation (i.e., rain, melted snow)
    \item Snowfall
    \item Snow depth
    \item Other elements where available
\end{itemize}

We use an alternate form of the GHCN Daily dataset. The period of record station files are parsed into  
yearly files that contain all available GHCN Daily station data for that year 
plus a time of observation field (where available--primarily for U.S. Cooperative 
Observers).  The obsertation times for U.S. Cooperative Observer data 
come from the station histories archived in NCDC's Multinetwork Metadata System (MMS).  

The yearly files are formatted so that every observation 
(i.e.,station/year/month/day/element/observation time) is represented by a single row 
with the following fields:
\begin{itemize}
	\item  station identifier (GHCN Daily Identification Number)
 	\item date (yyyymmdd; where yyyy=year; mm=month; and, dd=day)
 	\item observation type (see ftp://ftp.ncdc.noaa.gov/pub/data/ghcn/daily/readme.txt for definitions)
 	\item observation value (see ftp://ftp.ncdc.noaa.gov/pub/data/ghcn/daily/readme.txt for units)
 	\item observation time (if available, as hhmm where hh=hour and mm=minutes in local time)
 \end{itemize}
 
Sample data:

US1FLSL0019,20090101,PRCP,0,,,N,
 
US1FLSL0019,20090101,SNOW,0,,,N,

ASN00037003,20090101,PRCP,150,,,a,

USC00178998,20090101,TMAX,-111,,,0,1800

USC00178998,20090101,TMIN,-183,,,0,1800

USC00178998,20090101,TOBS,-139,,,0,1800

 \subsection{Data Extraction}
 
 For milestone $2$, we took the year file of 2013, extracted the five key elements, and collected by station identifiers. 
 

{\color{red} Menne, M.J., I. Durre, R.S. Vose, B.E. Gleason, and T.G. Houston, 2012:  An overview 
of the Global Historical Climatology Network-Daily Database.  Journal of Atmospheric 
and Oceanic Technology, 29, 897-910, doi:10.1175/JTECH-D-11-00103.1.}